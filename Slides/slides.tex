\documentclass[mathserif, 10pt]{beamer}
\usepackage[utf8]{inputenc}
\usepackage{amsmath, amsfonts}
\usepackage{appendixnumberbeamer}


\title{Tutorial: python y flavio}
\subtitle{Jorge Alda,\\ Universidad de Zaragoza/CAPA\hspace{4em} \texttt{jalda@unizar.es} }
\author[Jorge Alda]{Slides Example}

\date{Mayo 2022}



\usetheme{Zaragoza}
\usecolortheme{Unizar}
\titlepagelogoA{\includegraphics[width=6cm]{logos/CAPA.png}}
\titlepagelogoB{\includegraphics[width=4cm]{logos/dftuz2.png}}


\newcommand\colorcite[1]{{\scriptsize\color{blue}#1}}

\begin{document}
\begin{frame}[noframenumbering,plain]

    \titlepage

\end{frame}


\begin{frame}\frametitle{venv}
    Un ambiente virtual (venv) permite ``encapsular'' los paquetes de \texttt{python} que necesitas de forma aislada del resto del sistema operativo.

    ~

    Para crear un venv en el directorio actual:
    \begin{block}{Linux}
        \texttt{python3 -m venv envname}
    \end{block}
    \begin{block}{Windows}
        \texttt{C:\textbackslash Path\textbackslash to\textbackslash python -m venv envname}
    \end{block}

    Esto crea un subdirectorio \texttt{envname} con los archivos necesarios para usar el \texttt{venv}.
\end{frame}

\begin{frame}\frametitle{venv}

    Ahora hay que activar el texttt{venv}
    \begin{block}{Linux}
        \texttt{source envname/bin/activate}
    \end{block}
    \begin{block}{Windows}
        \texttt{envname\textbackslash Scripts\textbackslash Activate.ps1}
    \end{block}

    Comprueba que estás dentro del \texttt{venv} usando el comando \texttt{python -m pip list}.

    Para salir del \texttt{venv}, usa el comando
    \begin{block}{Windows \& Linux}
        deactivate
    \end{block}
\end{frame}

\begin{frame}\frametitle{Preparando el venv}
    Activa el \texttt{venv}, e instala los siguientes paquetes usando \texttt{pip}:
    \begin{itemize}
        \item numpy
        \item scipy
        \item matplotlib
\item jupyter
        \item flavio
    \end{itemize}
\end{frame}

\begin{frame}\frametitle{git y GitHub}

    \begin{itemize}
        \item \texttt{git} es un sistema de control de versiones. Permite mantener una historia de todos los cambios de un proyecto de código y trabajar en paralelo con varias versiones (ramas).
              \begin{itemize}
                  \item Cliente de línea de comandos en \url{https://git-scm.com/downloads}
                  \item También hay clientes gráficos o integrados en IDE (VSCode).
              \end{itemize}
        \item \texttt{GitHub} es un servidor de git. Permite mantener una copia ``en la nube'' del proyecto y su historia, editar de forma colaborativa, y compartir el código (open software).
              \begin{itemize}
                  \item Crear una cuenta en \url{https://github.com/}
                  \item Cliente de línea de comandos en \url{https://github.com/cli/cli}
              \end{itemize}
        \item Conecta \texttt{git} con la cuenta de \texttt{GitHub} con el comando \texttt{gh auth login} y sigue las instrucciones (cuenta de \texttt{GitHub} básica, protocolo HTTPS, autenticación usando browser).
    \end{itemize}
\end{frame}

\begin{frame}\frametitle{Empezando con git}
    \begin{enumerate}
        \item Entra en github.com con tu cuenta, pulsa el botón + en la esquina superior derecha y crea un nuevo repositorio. El nombre del repositorio es test-github, y añade un README.
        \item En la página del repositorio, pulsa el botón verde ``Code'' y copia la dirección. En la consola de comandos, navega al directorio donde quieras crear tu proyecto y ejecuta \texttt{git clone https://github.com/your-name/test-git.git}. El archivo README.md se habrá copiado a tu ordenador.
\end{enumerate}
\end{frame}

\begin{frame}\frametitle{Haciendo cambios en git}
\begin{enumerate}\setcounter{enumi}{3}
\item Crea un archivo test.txt en el directorio y guárdalo.
\item Ejecuta \texttt{git status}. Verás que el archivo test.txt ha sido modificado desde la última entrada en la historia del repositorio.
\item Ejecuta \texttt{git add test.txt}. Esto añade el archivo a la lista de preparación (stage), pero aún no a la historia.
\item Vuelve a ejecutar \texttt{git status}, el archivo está en el stage.
\item Una vez que hayas preparado todos los archivos modificados, puedes añadir una nueva entrada (commit) a la historia local con el comando \texttt{git commit -m "Descripción del commit"} (intenta ser descriptivo).
\item Puedes ver todos los commits locales con \texttt{git log}.
\end{enumerate}
    

\end{frame}

\begin{frame}\frametitle{Fusionar ramas}

    Puedes comparar los contenidos de dos ramas, línea por línea, con el comando \texttt{git diff main rama1}.

    ~

    Para fusionar dos ramas, git hace una comparación a 3, entre el último commit de cada rama y el commit en el que sus historias se bifurcaron. En las líneas de código en las que hay una coincidencia de al menos dos de las fuentes, git se queda con la versión que no está en el origen de la bifurcación. Si las tres fuentes discrepan, hay un conflicto de fusión, que hay que solucionar a mano.

    ~

    Para fusionar dos ramas, activa la rama que recibirá la fusión (en este caso main), y ejecuta \texttt{git merge rama1}

\end{frame}

\end{document}
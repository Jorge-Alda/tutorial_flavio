\documentclass[mathserif, 10pt]{beamer}
\usepackage[utf8]{inputenc}
\usepackage{amsmath, amsfonts}
\usepackage{appendixnumberbeamer}


\title{Tutorial: python y flavio}
\subtitle{Jorge Alda,\\ Universidad de Zaragoza/CAPA\hspace{4em} \texttt{jalda@unizar.es} }
\author[Jorge Alda]{Slides Example}

\date{Mayo 2022}



\usetheme{Zaragoza}
\usecolortheme{Unizar}
\titlepagelogoA{\includegraphics[width=6cm]{logos/CAPA.png}}
\titlepagelogoB{\includegraphics[width=4cm]{logos/dftuz2.png}}


\newcommand\colorcite[1]{{\scriptsize\color{blue}#1}}

\begin{document}
\begin{frame}[noframenumbering,plain]

    \titlepage

\end{frame}


\begin{frame}\frametitle{venv}
    Un ambiente virtual (venv) permite ``encapsular'' los paquetes de \texttt{python} que necesitas de forma aislada del resto del sistema operativo.

    ~

    Para crear un venv en el directorio actual:
    \begin{block}{Linux}
        \texttt{python3 -m venv envname}
    \end{block}
    \begin{block}{Windows}
        \texttt{C:\textbackslash Path\textbackslash to\textbackslash python -m venv envname}
    \end{block}

    Esto crea un subdirectorio \texttt{envname} con los archivos necesarios para usar el \texttt{venv}.
\end{frame}

\begin{frame}\frametitle{venv}

    Ahora hay que activar el texttt{venv}
    \begin{block}{Linux}
        \texttt{source envname/bin/activate}
    \end{block}
    \begin{block}{Windows}
        \texttt{envname\textbackslash Scripts\textbackslash Activate.ps1}
    \end{block}

    Comprueba que estás dentro del \texttt{venv} usando el comando \texttt{python -m pip list}.

    Para salir del \texttt{venv}, usa el comando
    \begin{block}{Windows \& Linux}
        deactivate
    \end{block}
\end{frame}

\begin{frame}\frametitle{Preparando el venv}
    Activa el \texttt{venv}, e instala los siguientes paquetes usando \texttt{pip}:
    \begin{itemize}
        \item numpy
        \item scipy
        \item matplotlib
        \item flavio
    \end{itemize}
\end{frame}

\end{document}